\documentclass[12pt, aip, jcp]{revtex4-2} % Using a common physics journal class
\usepackage{amsmath,amssymb,geometry}
\usepackage[utf8]{inputenc}
\usepackage{hyperref}

\geometry{a4paper, margin=1in}

\begin{document}

\title{Cosmic Thermodynamics as a Signature of Universal Consciousness:\\A Falsifiable Prediction from the Recognition Science Framework}

\author{Jonathan Washburn}
\email{washburn@recognitionphysics.org}
\affiliation{Recognition Science Institute, Austin, Texas, USA}

\date{\today}

\begin{abstract}
\noindent This paper presents a novel, falsifiable prediction for the total number of conscious observers in the universe, derived from the axioms of the Recognition Science (RS) framework. We posit that the thermodynamics of the cosmos, specifically the temperature of the cosmic microwave background (CMB), are inextricably linked to the computational heat generated by the sum of all conscious recognition events. The derivation connects Landauer's principle of irreversible computation to the RS cost functional, calculating the irreducible power consumption of a single conscious agent. By equating the total heat generated by all such agents to the universe's black-body cooling capacity, we derive a "Master Equation of Cosmic Thermodynamics." This equation yields a concrete prediction that the universe can sustain approximately \(10^{68}\) conscious entities. This work transforms a philosophical question into a quantitative, testable hypothesis, linking cosmology, information theory, and consciousness within a single, rigid framework. We conclude by discussing the model's primary assumptions and outlining clear avenues for its falsification.
\end{abstract}

\maketitle

% The main body of the paper will follow here, starting with Section 1: Introduction.

\end{document}
